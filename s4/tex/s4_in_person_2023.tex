\documentclass{article}

\usepackage{amsmath}
\usepackage{amsthm}
\usepackage{amssymb}
\usepackage{bbm}
\usepackage{fancyhdr}
\usepackage{listings}
\usepackage{cite}
\usepackage{graphicx}
\usepackage{enumitem}
\usepackage[margin=1cm]{caption}
\usepackage{subcaption}
\usepackage{tcolorbox}
\usepackage{color}
\definecolor{editorGray}{rgb}{0.95, 0.95, 0.95}
\usepackage{xcolor}

\lstset{%
    % Basic design
    backgroundcolor=\color{editorGray},
    basicstyle={\small\ttfamily},   
    frame=l,
    % Line numbers
    xleftmargin={0.75cm},
    numbers=left,
    stepnumber=1,
    firstnumber=1,
    numberfirstline=false,
    }
\lstset{
    literate={~} {$\sim$}{1}
}

\newenvironment{claim}[1]{\par\noindent\underline{Claim:}\space#1}{}
\newenvironment{claimproof}[1]{\par\noindent\underline{Proof:}\space#1}{\hfill $\blacksquare$}

\oddsidemargin 0in \evensidemargin 0in
\topmargin -0.5in \headheight 0.25in \headsep 0.25in
\textwidth 6.5in \textheight 9in
% set the header up
\fancyhead{}
\fancyhead[L]{Stanford University}
\fancyhead[R]{Principles of Robot Autonomy I - Fall 2023}

%%%%%%%%%%%%%%%%%%%%%%%%%%
\renewcommand\headrulewidth{0.4pt}
\setlength\headheight{15pt}
\input{preamble}

\usepackage{outlines}

\usepackage{gensymb}
\usepackage{algorithm}
\usepackage{algpseudocode}


\newcommand{\ssmargin}[2]{{\color{blue}#1}{\marginpar{\color{blue}\raggedright\scriptsize [SS] #2 \par}}}

\newcommand{\x}{\mathbf{x}}

\setlength{\parindent}{0in}

\title{AA 274A: Principles of Robotic Autonomy I \\Section 4 (in-person): Controller Gain Tuning in Hardware}
\date{}

\setlength{\textfloatsep}{10pt} % Change vertical space after algorithm block


\begin{document}

\maketitle
\pagestyle{fancy}
\vspace{-1.25cm}
% Our goals for this section: 
% \begin{enumerate}
%     \item Learn how to read and understand source code for more complex ROS nodes
%     \item Test controllers from homework on a real robot
%     \item Learn how to design custom launch files
% \end{enumerate}
\section{Setup}
% students should have a git repo with their section group that they develop
% \color{blue}
{\bf Task 1.1 -- Cleanup.} Before starting this section, remove any existing files in \texttt{~/autonomy\_ws/src/} on the AA274 laptop.

{\bf Task 1.2 -- Git clone your repo.} Clone your group's workspace repository into the \texttt{~/autonomy\_ws/src/} directory and build the workspace:
\begin{lstlisting}
cd ~/autonomy_ws/src/
git clone <your repo>
colcon build
\end{lstlisting}

{\bf Task 1.3 -- Copy homework controller.} We will be using the code you wrote in HW1 to control the turtlebot. Copy the file {\color{blue} \texttt{heading\_controller.py}} to the folder \texttt{~/autonomy\_ws/src/scripts} on the AA274 laptop. 

\section{Running the Controller}
* Make sure to proceed with code that has successfully accomplished controller tasks in gazebo for the homework. 

{\bf Task 2.1 -- Robot launch.} Place the turtlebot on the ground, and switch on. In a terminal on the AA274 laptop, ssh into the robot and run,
\begin{lstlisting}
micromamba activate ros_env
ros2 launch asl-tb3-driver bringup.launch.py
\end{lstlisting}

{\bf Task 2.2 -- Laptop launch.}  In a new terminal navigate to \texttt{~/autonomy\_ws/}, and run
\begin{lstlisting}
export ROS_DOMAIN_ID=<Your Robot's ID>
source install/local_setup.bash
ros2 launch autonomy_ws heading_controller.launch.py use_sim_time:=false
\end{lstlisting}

{\bf Task 2.3 -- Control to goal.} Now you can specify goal poses using the ``2D Nav Goal" button in rviz and clicking and dragging on the map. The robot should move towards the goal if your controllers work correctly!

\section{ROS Parameters}
{\bf Task 3.1 -- List ROS parameters.} With robot and laptop launch files still active, run
\begin{lstlisting}
ros2 param list
\end{lstlisting}

{\bf Task 3.2 -- Identify yaw proportional control.}

\section{Visualizing the Goal}
Using what you learned in last section, write a new node that visualizes the current navigation target in rviz as a marker. Save this node in the \texttt{section5} package's \texttt{scripts} folder.

\textbf{Problem 5: Describe at a high level how your goal visualizer works. Some questions to get you started are: 
\begin{itemize}
    \item What topics should it subscribe to in order to stay up to date with the current navigation target?
    \item What message type should it publish, and to what topic?
\end{itemize} 
Include this code in your submission.}


\end{document}
