\documentclass{article}

\usepackage{amsmath}
\usepackage{amsthm}
\usepackage{amssymb}
\usepackage{bbm}
\usepackage{fancyhdr}
\usepackage{listings}
\usepackage{cite}
\usepackage{graphicx}
\usepackage{enumitem}
\usepackage[margin=1cm]{caption}
\usepackage{subcaption}
\usepackage{tcolorbox}
\usepackage{url}
\usepackage{hyperref}

\definecolor{editorGray}{rgb}{0.95, 0.95, 0.95}

\hypersetup{
    colorlinks=true,
    linkcolor=blue,
    filecolor=magenta,      
    urlcolor=blue,
}

\lstset{%
    % Basic design
    backgroundcolor=\color{editorGray},
    basicstyle={\small\ttfamily},   
    frame=l,
    % Line numbers
    xleftmargin={0.75cm},
    numbers=left,
    stepnumber=1,
    firstnumber=1,
    numberfirstline=false,
    }
    
\lstset{
    literate={~} {$\sim$}{1}
}

\newenvironment{claim}[1]{\par\noindent\underline{Claim:}\space#1}{}
\newenvironment{claimproof}[1]{\par\noindent\underline{Proof:}\space#1}{\hfill $\blacksquare$}

\oddsidemargin 0in \evensidemargin 0in
\topmargin -0.5in \headheight 0.25in \headsep 0.25in
\textwidth 6.5in \textheight 9in
\parskip 6pt \parindent 0in \footskip 20pt

% set the header up
\fancyhead{}
\fancyhead[L]{Stanford Aeronautics \& Astronautics}
\fancyhead[R]{Fall 2021}

%%%%%%%%%%%%%%%%%%%%%%%%%%
\renewcommand\headrulewidth{0.4pt}
\setlength\headheight{15pt}
\input{preamble}

\usepackage{outlines}

\usepackage{xparse}
\NewDocumentCommand{\codeword}{v}{%
\texttt{\textcolor{blue}{#1}}%
}
\usepackage{gensymb}

\newcommand{\ssmargin}[2]{{\color{blue}#1}{\marginpar{\color{blue}\raggedright\scriptsize [SS] #2 \par}}}


\setlength{\parindent}{0in}

\title{AA 274A: Principles of Robot Autonomy I \\ Section 2: Workstation and ROS}
\date{}

\begin{document}

\maketitle
\pagestyle{fancy}

Our goals for this section:
\begin{enumerate}
    \item Learn the basic functionality of the Robot Operating System (ROS). 
    \item Create a basic ROS package, and understand file structure.
    \item Implement a basic ROS node that sends constant commands to a Turtlebot.
\end{enumerate}

TODO: SECTION GIT INSTRUCTIONS HERE
\begin{lstlisting}
GIT COMMANDS
\end{lstlisting}

\section{Creating a ROS workspace}
A ROS Development Workspace is a directory that is used to organize packages that extend the basic functionality of ROS. Use the following commands to create a ROS workspace.

\begin{lstlisting}
    mkdir tb_ws         # Make a new directory ~/tb_ws
    cd tb_ws            # Change into that directory 
    mkdir src           # Make a new sub-directory ~/tb_ws/src
\end{lstlisting}

Now that we have a basic file structure we can use ROS's build tool, {\it colcon}, to generate the rest of the core workspace files and folders. 

\begin{lstlisting}
    colcon build        # This command should be run in ~/tb_ws
\end{lstlisting}

After running that command you should see a message ``\texttt{Summary: 0 packages finished}", and should have some new folders in your workspace (you can use \texttt{ls} to verify this).



\section{Creating a basic ROS Package}
ROS is a complex software-framework, and building ROS code can be difficult. To solve this problem ROS code is typically organized into packages, which are directories that contain a specific set of files and sub-directories. When code is organized into a package it can be built with ROS's build tool, {\it colcon}. Building a package links the local code to external libraries and other ROS infrastructure, complies any C++ code (for this course we'll stick to python), and creates executables for python code. 
\begin{lstlisting}
    
\end{lstlisting}


\end{document}