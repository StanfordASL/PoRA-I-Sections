\documentclass{article}

\usepackage{amsmath}
\usepackage{hyperref}
\usepackage{amsthm}
\usepackage{amssymb}
\usepackage{bbm}
\usepackage{fancyhdr}
\usepackage{listings}
\usepackage{cite}
\usepackage{graphicx}
\usepackage{enumitem}
\usepackage[margin=1cm]{caption}
\usepackage{subcaption}
\usepackage{tcolorbox}
\usepackage{color}
\definecolor{editorGray}{rgb}{0.95, 0.95, 0.95}
\hypersetup{
  colorlinks=true,
  linkcolor=blue,
  filecolor=magenta,
  urlcolor=blue,
}

\lstset{%
  % Basic design
  backgroundcolor=\color{editorGray},
  basicstyle={\small\ttfamily},
  commentstyle=\color{teal},
  keywordstyle=\color{orange},
  stringstyle=\color{violet},
  showstringspaces=false,
  showspaces=false,
  frame=single,
  % Line numbers
  xleftmargin={0.75cm},
  numbers=left,
  stepnumber=1,
  firstnumber=1,
}
\lstset{
  literate={~} {$\sim$}{1}
}

\newenvironment{claim}[1]{\par\noindent\underline{Claim:}\space#1}{}
\newenvironment{claimproof}[1]{\par\noindent\underline{Proof:}\space#1}{\hfill $\blacksquare$}

\oddsidemargin 0in \evensidemargin 0in
\topmargin -0.5in \headheight 0.25in \headsep 0.25in
\textwidth 6.5in \textheight 9in
\parskip 6pt \parindent 0in \footskip 20pt

% set the header up
\fancyhead{}
\fancyhead[L]{Stanford Aeronautics \& Astronautics}
\fancyhead[R]{Autumn 2023 - 2024}

%%%%%%%%%%%%%%%%%%%%%%%%%%
\renewcommand\headrulewidth{0.4pt}
\setlength\headheight{15pt}

\usepackage{outlines}

\usepackage{xparse}
\def\code#1{\texttt{\color{magenta} #1}}
\usepackage{gensymb}

\newcommand{\ssmargin}[2]{{\color{blue}#1}{\marginpar{\color{blue}\raggedright\scriptsize [SS] #2 \par}}}


\setlength{\parindent}{0in}

\title{Principles of Robotic Autonomy \\ Section 1: UNIX, Git, Python}
\date{}

\begin{document}

\maketitle
\pagestyle{fancy}

Our goals for this section: \begin{enumerate}
  \item Learn to navigate in a UNIX OS with terminal commands.
  \item Learn to use Git to create and track software development.
  \item Learn to write Python and Shell executables.
\end{enumerate}

\section{Working with UNIX Terminal}

Complete the following tasks using terminal commands only. Don't use the graphical
interface and save those commands (in your mind) as you will need them later.

{\bf Task 1.1 -- Cleanup.} Remove directory \code{$\sim$/autonomy\_ws} if they exist. This
might come from a previous section.

{\bf Task 1.2 -- Create workspace.} Pick a name for the directory that would host your robot
autonomy code in the future. For the reset of this document, we use \code{<repo>} to
denote the directory name you chose. Create a nested directory at
\code{$\sim$/autonomy\_ws/src/<repo>}.

{\bf Task 1.3 -- Add a README.} Create a \code{README.md} file in \code{<repo>} you
just created, and add a title to it. For example
\begin{lstlisting}[numbers=none]
# We Build Autonomous Robots
\end{lstlisting}
Hint: you may use your favorite code editing software to edit the README file.
For example, you can launch VSCode to edit any file with the following command,
\begin{lstlisting}[language=sh, numbers=none]
code <path>/<to>/<file>
\end{lstlisting}
Other code editing software works as well, e.g. \code{vim}, \code{nano}, \code{emacs}, etc.

{\bf Cheat Sheets}
\begin{enumerate}
\item {\bf Special Directories}
\begin{lstlisting}[language=sh]
~   # home directory
.   # current directory
..  # parent directory
/   # root directory
\end{lstlisting}

\item {\bf File System Commands}
\begin{lstlisting}[language=sh]
pwd             # print out the current working directory
ls              # list all files and sub-directories under pwd
ls -l <file>    # show permissions (read, write, executable) of <file>
cd <dir>        # change current directory to <dir>
mkdir <dir>     # create a single directory at <dir>
mkdir -p <dir>  # create a possibly nested directory at <dir>
mv <A> <B>      # rename file <A> to file <B>
cp <A> <B>      # copy file <A> to file <B>
rm <file>       # remove a single <file>
rm -r <dir>     # remove a directory <dir> and everything in it
rm -rf <dir>    # same as above but also work if <dir> does not exist
touch <file>    # create an empty <file>
chmod +x <file> # make <file> executable
chmod -w <file> # make <file> read-only
cat <file>      # print all the content of <file>
du -sh <dir>    # show total size of directory <dir>
\end{lstlisting}
\end{enumerate}

\section{Using Git}
Git is a source control tool that allows people to share code with each other, and more
importantly, keep track of a history of code changes for any type of software development.
We will guide you through creating a Git repository and share with you some common workflows
when working collaboratively with multiple people on the same repository.

{\bf Task 2.1 -- Create a GitHub account.} Go to \url{https://github.com} and create an
account with your university email if you have not yet done so before.

{\bf Task 2.2 -- GitHub authentication.} Use the following command and follow the prompts
to authenticate the terminal with your GitHub account.
\begin{lstlisting}[language=sh, numbers=none]
gh auth login -w
\end{lstlisting}

{\bf Task 2.3 -- Setup a GitHub repository.} Create a \boxed{\textbf{private}}
new repository with the name \code{<repo>} you picked earlier. Follow the guide on
you newly created GitHub project page to initialize \code{$\sim$/autonomy\_ws/src/<repo>}
as a git repository and push the README to GitHub.

{\bf Task 2.4 -- Add collaborators.} Go to project setting in GitHub and add your teammates
as collaborators to the repository you just created.

{\bf Task 2.5 -- Add group info by creating a PR.} Use the following steps as a guide
\begin{enumerate}
  \item Create a new branch.
  \item Add a new file named \code{team.txt} and edit it to include all SUNetIDs of
    your team members (separated by newlines).
  \item Commit and push the new file to the new branch.
  \item Create a pull request from the GitHub project page
  \item Review the changes and merge it back to \code{main} branch.
\end{enumerate}

{\bf Cheat Sheets}
\begin{enumerate}
\item {\bf Generic.}
\begin{lstlisting}[language=sh]
git fetch             # sync remote branch (does not change local files)
git pull              # sync remote branch and also update local files
git clone <url>       # clone git repo from <url> to the current directory
git checkout <branch> # switch to <branch>
git add .             # add all files in the current directory
git add <file or dir> # add a single file or directory
git add -u            # add all tracked files
git commit -m "<msg>" # commit all added changes to the local branch
\end{lstlisting}

\item {\bf On \code{main} branch.}
\begin{lstlisting}[language=sh]
git push                  # push commited changes to main branch
git checkout -b <branch>  # create and switch to a new branch
\end{lstlisting}

\item {\bf On a feature branch.}
\begin{lstlisting}[language=sh]
git push -u origin <branch> # 1st push on a newly created local branch
git merge origin/main       # sync with the latest main branch
\end{lstlisting}
\end{enumerate}

\section{Executables}
All executables are just files with the \code{x} permission turned on. Here
we focus on two scripting languages, Python and Shell, which you will use the most
to build up your robot autonomy stack.

\subsection{Python}
Python is the main programming language for this course. In this section, we will
guide you through how to properly create a Python executable. You will need to repeat
the same process later when creating your own ROS2 nodes.

{\bf Task 3.1.1 -- Create a Python executable.}

\begin{enumerate}
\item Create a Python file at \code{<repo>/scripts/section1.py}.

\item Edit the file and use the following code as a start-up template
\begin{lstlisting}[language=Python]
#!/usr/bin/env python3

# add import and helper functions here

if __name__ == "__main__":
    # code goes here
\end{lstlisting}

The first line, a.k.a the shebang, is always required for the OS to find the correct
Python interpreter to execute this file. Shell scripts will require a different
shebang (see Section~\ref{sec:shell}).

\item Change the permission of the file to make it an executable.

    Hint: use \code{chmod}.

\item Check that the permission of the file is indeed correct by running the following command,
\begin{lstlisting}[language=sh, numbers=none]
ls -l <path to section1.py>
\end{lstlisting}
and see that the \code{x} permission is indeed turned on for the file.

\end{enumerate}

{\bf Task 3.1.2 -- Execute it!} Add something simple to the file
(e.g. \code{print("hello world")}), and then try to execute the file by
writing the full path to \code{section1.py} and hit return.

Note: if you are currently in the \code{<repo>/scripts} directory in your terminal,
you cannot just type \code{section1.py} and run the code. Instead, you need to
type \code{./section1.py}.

{\bf Task 3.1.3 -- Learn about \code{numpy} matmul operator}
Import numpy as \code{np} in your Python script and
add the following snippet to the main block to generate two random matrices.
\begin{lstlisting}[language=Python]
np.random.seed(42)
A = np.random.normal(size=(4, 4))
B = np.random.normal(size=(4, 2))
\end{lstlisting}
Write code to compute and print out the matrix product $A \cdot B$
using the Python matmul operator -- \code{@}. The result should match the following
\begin{lstlisting}[language=Python, numbers=none]
array([[ 0.67459114, -1.96480447],
       [ 2.81615463, -1.19285965],
       [-0.72781678, -0.14561428],
       [-1.07385636,  3.96873247]])
\end{lstlisting}

{\bf Task 3.1.3 -- Learn about \code{numpy} slicing and broadcasting.} Add the following
snippet to the main block to generate a batch of size-10 vectors.
\begin{lstlisting}[language=Python]
np.random.seed(42)
x = np.random.normal(size=(4, 10))
\end{lstlisting}
Write code to compute for a L2 distance matrix among the vectors. Let $D$ denote
such matrix, then it can be defined with
\begin{gather}
  D_{ij} = \|x_i - x_j\|_2.
\end{gather}
\textbf{Do NOT use for loops}.

The results should match the following
\begin{lstlisting}[language=Python, numbers=none]
array([[0.        , 5.12541489, 4.30214487, 4.95259829],
       [5.12541489, 0.        , 2.69591513, 4.74784078],
       [4.30214487, 2.69591513, 0.        , 4.52181487],
       [4.95259829, 4.74784078, 4.52181487, 0.        ]])
\end{lstlisting}

Hints:
\begin{enumerate}
  \item \code{np.linalg.norm} could be useful.
  \item Some intermediate variable broadcasts to a shape of (4, 4, 10).
  \item (Optional) A clever math trick can reduce the broadcast size to (4, 4).
\end{enumerate}

\subsection{Shell}\label{sec:shell}
See \href{https://en.wikipedia.org/wiki/Shell_script}{Wikipedia} for some histories
of the shell language. In short, this is the default language to interface with most
UNIX terminal environments.

{\bf Task 3.2.1 -- Create a cleanup script.} Create a shell script at
\code{<repo>/scripts/cleanup.sh} and change the permission to make it an executable.
You also need to add the following shebang to the top of the script.
\begin{lstlisting}[language=sh, numbers=none]
#!/usr/bin/sh
\end{lstlisting}

{\bf Task 3.2.2 -- Write the cleanup script.} This script will be used to quickly cleanup
your workspace at the end of each section. Edit \code{cleanup.sh} to achieve
the following functionalities:

\begin{enumerate}
\item Logout your GitHub account -- you can add the following line to your shell script
\begin{lstlisting}[language=sh, numbers=none]
gh auth logout
\end{lstlisting}

\item Remove your workspace directory \code{$\sim$/autonomy\_ws}.
\end{enumerate}

\section{Cleanup}

{\bf Task 4.1 -- Upload changes to GitHub.} Commit and push all changes to a new branch
and create a PR and name it ``Section 1''.

{\bf Task 4.2 -- Cleanup.} Run \code{cleanup.sh} to clean up
your workspace for the next section.

\end{document}
